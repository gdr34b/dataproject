% This is "sig-alternate.tex" V2.0 May 2012
% This file should be compiled with V2.5 of "sig-alternate.cls" May 2012
%
% This example file demonstrates the use of the 'sig-alternate.cls'
% V2.5 LaTeX2e document class file. It is for those submitting
% articles to ACM Conference Proceedings WHO DO NOT WISH TO
% STRICTLY ADHERE TO THE SIGS (PUBS-BOARD-ENDORSED) STYLE.
% The 'sig-alternate.cls' file will produce a similar-looking,
% albeit, 'tighter' paper resulting in, invariably, fewer pages.
%
% ----------------------------------------------------------------------------------------------------------------
% This .tex file (and associated .cls V2.5) produces:
%       1) The Permission Statement
%       2) The Conference (location) Info information
%       3) The Copyright Line with ACM data
%       4) NO page numbers
%
% as against the acm_proc_article-sp.cls file which
% DOES NOT produce 1) thru' 3) above.
%
% Using 'sig-alternate.cls' you have control, however, from within
% the source .tex file, over both the CopyrightYear
% (defaulted to 200X) and the ACM Copyright Data
% (defaulted to X-XXXXX-XX-X/XX/XX).
% e.g.
% \CopyrightYear{2007} will cause 2007 to appear in the copyright line.
% \crdata{0-12345-67-8/90/12} will cause 0-12345-67-8/90/12 to appear in the copyright line.
%
% ---------------------------------------------------------------------------------------------------------------
% This .tex source is an example which *does* use
% the .bib file (from which the .bbl file % is produced).
% REMEMBER HOWEVER: After having produced the .bbl file,
% and prior to final submission, you *NEED* to 'insert'
% your .bbl file into your source .tex file so as to provide
% ONE 'self-contained' source file.
%
% ================= IF YOU HAVE QUESTIONS =======================
% Questions regarding the SIGS styles, SIGS policies and
% procedures, Conferences etc. should be sent to
% Adrienne Griscti (griscti@acm.org)
%
% Technical questions _only_ to
% Gerald Murray (murray@hq.acm.org)
% ===============================================================
%
% For tracking purposes - this is V2.0 - May 2012

\documentclass{sig-alternate}

\begin{document}
%
% --- Author Metadata here ---
% \conferenceinfo{WOODSTOCK}{'97 El Paso, Texas USA}
% \CopyrightYear{2007} % Allows default copyright year (20XX) to be over-ridden - IF NEED BE.
% \crdata{0-12345-67-8/90/01}  % Allows default copyright data (0-89791-88-6/97/05) to be over-ridden - IF NEED BE.
% --- End of Author Metadata ---

\title{On Building A Data Fitting System Using Ad Hoc Models}

%
% You need the command \numberofauthors to handle the 'placement
% and alignment' of the authors beneath the title.
%
% For aesthetic reasons, we recommend 'three authors at a time'
% i.e. three 'name/affiliation blocks' be placed beneath the title.
%
% NOTE: You are NOT restricted in how many 'rows' of
% "name/affiliations" may appear. We just ask that you restrict
% the number of 'columns' to three.
%
% Because of the available 'opening page real-estate'
% we ask you to refrain from putting more than six authors
% (two rows with three columns) beneath the article title.
% More than six makes the first-page appear very cluttered indeed.
%
% Use the \alignauthor commands to handle the names
% and affiliations for an 'aesthetic maximum' of six authors.
% Add names, affiliations, addresses for
% the seventh etc. author(s) as the argument for the
% \additionalauthors command.
% These 'additional authors' will be output/set for you
% without further effort on your part as the last section in
% the body of your article BEFORE References or any Appendices.

\numberofauthors{3} %  in this sample file, there are a *total*
% of EIGHT authors. SIX appear on the 'first-page' (for formatting
% reasons) and the remaining two appear in the \additionalauthors section.
%
\author{
% You can go ahead and credit any number of authors here,
% e.g. one 'row of three' or two rows (consisting of one row of three
% and a second row of one, two or three).
%
% The command \alignauthor (no curly braces needed) should
% precede each author name, affiliation/snail-mail address and
% e-mail address. Additionally, tag each line of
% affiliation/address with \affaddr, and tag the
% e-mail address with \email.
%
\alignauthor
    Amy Briggs \\
   \email{abbr5@mst.edu}
\alignauthor
    Andrew Fallgren \\
    \email{ajffk6@mst.edu}
\alignauthor
    George Rush \\
    \email{gdr34b@mst.edu}
}

\maketitle
\begin{abstract}
One class of data is measured or simulated data with error estimation. This data can consist of many continuous dimensions for which values are available only at discrete points. Increasing the number of discrete points at which the data is available can be expensive or even impossible to obtain, but it can still be useful for predicting data trends. Unfortunately, this is difficult when the various dimensions do not follow the same type of fit (linear, logarithmic, polynomial, etc.). Our approach focuses on building decision trees and using them to interpolate new data points that follow existing trends. This is in contrast to previous methods which focused on extrapolating data for specific applications or using purely numerical regression models. By using this approach, sparse data sets or those that exhibit unusual patterns can be analyzed effectively.
\end{abstract}

\category{H.2.8}{Database Management}{Database Applications}[data mining]

\terms{Algorithms}

\keywords{data mining, sparse data, interpolation}

\section{Introduction}
Outline goes here.
\begin{itemize}
    \item The first item
    \item The second item
    \item The third etc \ldots
\end{itemize}
\subsection{Stuff}
This is a subsection.
\subsubsection{More Stuff}
This is a subsubsection.

\section{Related Work}
Another section. I am citing something random \cite{bowman:reasoning}.

\section{Methodology}
\subsection{Decision Tree Interpolation}
Decision Tree Interpolation follows this process:
\begin{enumerate}
    \item Build a decision tree from the original data.
    \item For each leaf node in the tree:
    \begin{enumerate}
        \item Obtain all attribute values for associated data instances.
        \item Define ranges for attribute values.
        \begin{itemize}
            \item For numeric attributes, define the range using minimum and maximum values.
            \item For discrete attributes, define the range as all distinct values.
        \end{itemize}
        \item Calculate the distribution of all associated data instances.
        \item Create new data points within the ranges that match the statistical distribution.
    \end{enumerate}
\end{enumerate}
Note that the number of data points created per leaf node is proportional to the number of data points already classified by that leaf node. This ensures that any interpolated data will follow the overall data distribution, at least relative to the data density per leaf node. 

\subsection{Interpolated Data Validation}
All interpolated data is validated through this process:
\begin{enumerate}
    \item A new decision tree is built based on the interpolated data. Note that the original data is \textit{not} included here.
    \item Both the new and original decision trees are compared for accuracy against the new and original data sets.
\end{enumerate}
Note that any decision tree with an arbitrarily large maximum depth can classify data with perfect accuracy. Defining a low maximum depth means that classification is imperfect, and it is under these conditions that differences in the quality of different decision trees become apparent.

\subsection{Experiment Procedure}
Details about experiment variables go here.

\section{Results}
Another section.

\section{Discussion}
Another section.

\section{Conclusion and Future Work}
Last section.
%\end{document}  % This is where a 'short' article might terminate

%
% The following two commands are all you need in the
% initial runs of your .tex file to
% produce the bibliography for the citations in your paper.
\bibliographystyle{abbrv}
\bibliography{references}  % references.bib is the name of the Bibliography in this case
% You must have a proper ".bib" file
%  and remember to run:
% latex bibtex latex latex
% to resolve all references
%
% ACM needs 'a single self-contained file'!
% That's all folks!
\end{document}
