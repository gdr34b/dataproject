% This is "sig-alternate.tex" V2.0 May 2012
% This file should be compiled with V2.5 of "sig-alternate.cls" May 2012
%
% This example file demonstrates the use of the 'sig-alternate.cls'
% V2.5 LaTeX2e document class file. It is for those submitting
% articles to ACM Conference Proceedings WHO DO NOT WISH TO
% STRICTLY ADHERE TO THE SIGS (PUBS-BOARD-ENDORSED) STYLE.
% The 'sig-alternate.cls' file will produce a similar-looking,
% albeit, 'tighter' paper resulting in, invariably, fewer pages.
%
% ----------------------------------------------------------------------------------------------------------------
% This .tex file (and associated .cls V2.5) produces:
%       1) The Permission Statement
%       2) The Conference (location) Info information
%       3) The Copyright Line with ACM data
%       4) NO page numbers
%
% as against the acm_proc_article-sp.cls file which
% DOES NOT produce 1) thru' 3) above.
%
% Using 'sig-alternate.cls' you have control, however, from within
% the source .tex file, over both the CopyrightYear
% (defaulted to 200X) and the ACM Copyright Data
% (defaulted to X-XXXXX-XX-X/XX/XX).
% e.g.
% \CopyrightYear{2007} will cause 2007 to appear in the copyright line.
% \crdata{0-12345-67-8/90/12} will cause 0-12345-67-8/90/12 to appear in the copyright line.
%
% ---------------------------------------------------------------------------------------------------------------
% This .tex source is an example which *does* use
% the .bib file (from which the .bbl file % is produced).
% REMEMBER HOWEVER: After having produced the .bbl file,
% and prior to final submission, you *NEED* to 'insert'
% your .bbl file into your source .tex file so as to provide
% ONE 'self-contained' source file.
%
% ================= IF YOU HAVE QUESTIONS =======================
% Questions regarding the SIGS styles, SIGS policies and
% procedures, Conferences etc. should be sent to
% Adrienne Griscti (griscti@acm.org)
%
% Technical questions _only_ to
% Gerald Murray (murray@hq.acm.org)
% ===============================================================
%
% For tracking purposes - this is V2.0 - May 2012

\documentclass{sig-alternate}

\usepackage[table]{xcolor}

\begin{document}
%
% --- Author Metadata here ---
% \conferenceinfo{WOODSTOCK}{'97 El Paso, Texas USA}
% \CopyrightYear{2007} % Allows default copyright year (20XX) to be over-ridden - IF NEED BE.
% \crdata{0-12345-67-8/90/01}  % Allows default copyright data (0-89791-88-6/97/05) to be over-ridden - IF NEED BE.
% --- End of Author Metadata ---

\title{On Building A Data Fitting System Using Ad Hoc Models}

%
% You need the command \numberofauthors to handle the 'placement
% and alignment' of the authors beneath the title.
%
% For aesthetic reasons, we recommend 'three authors at a time'
% i.e. three 'name/affiliation blocks' be placed beneath the title.
%
% NOTE: You are NOT restricted in how many 'rows' of
% "name/affiliations" may appear. We just ask that you restrict
% the number of 'columns' to three.
%
% Because of the available 'opening page real-estate'
% we ask you to refrain from putting more than six authors
% (two rows with three columns) beneath the article title.
% More than six makes the first-page appear very cluttered indeed.
%
% Use the \alignauthor commands to handle the names
% and affiliations for an 'aesthetic maximum' of six authors.
% Add names, affiliations, addresses for
% the seventh etc. author(s) as the argument for the
% \additionalauthors command.
% These 'additional authors' will be output/set for you
% without further effort on your part as the last section in
% the body of your article BEFORE References or any Appendices.

\numberofauthors{3} %  in this sample file, there are a *total*
% of EIGHT authors. SIX appear on the 'first-page' (for formatting
% reasons) and the remaining two appear in the \additionalauthors section.
%
\author{
% You can go ahead and credit any number of authors here,
% e.g. one 'row of three' or two rows (consisting of one row of three
% and a second row of one, two or three).
%
% The command \alignauthor (no curly braces needed) should
% precede each author name, affiliation/snail-mail address and
% e-mail address. Additionally, tag each line of
% affiliation/address with \affaddr, and tag the
% e-mail address with \email.
%
\alignauthor
    Amy Briggs \\
   \email{abbr5@mst.edu}
\alignauthor
    Andrew Fallgren \\
    \email{ajffk6@mst.edu}
\alignauthor
    George Rush \\
    \email{gdr34b@mst.edu}
}

\maketitle
\begin{abstract}
One class of data is measured or simulated data with error estimation. This data can consist of many continuous dimensions for which values are available only at discrete points. Increasing the number of discrete points at which the data is available can be expensive or even impossible to obtain, but it can still be useful for predicting data trends. Unfortunately, this is difficult when the various dimensions do not follow the same type of fit (linear, logarithmic, polynomial, etc.). Our approach focuses on building decision trees and using them to interpolate new data points that follow existing trends. This is in contrast to previous methods which focused on extrapolating data for specific applications or using purely numerical regression models. By using this approach, sparse data sets or those that exhibit unusual patterns can be analyzed effectively.
\end{abstract}

\category{H.2.8}{Database Management}{Database Applications}[data mining]

\terms{Algorithms}

\keywords{data mining, sparse data, interpolation}

\section{Introduction}
Outline goes here.
\begin{itemize}
    \item The first item
    \item The second item
    \item The third etc \ldots
\end{itemize}

Text text text text text text. Text text text text text text text. Text text text text text. Text text text text text text text text. Text text text text text text. Text text text text text text text. Text text text text text. Text text text text text text text text. Text text text text text text. Text text text text text text text. Text text text text text. Text text text text text text text text.

\subsection{Stuff}
This is a subsection.
\subsubsection{More Stuff}
This is a subsubsection.

Text text text text text text. Text text text text text text text. Text text text text text. Text text text text text text text text. Text text text text text text. Text text text text text text text. Text text text text text. Text text text text text text text text. Text text text text text text. Text text text text text text text. Text text text text text. Text text text text text text text text.

\section{Related Work}
Early works on the interpolation of scattered data evaluate a variety of different computational methods that focus on obtaining a smooth function $F(x, y)$ to follow the dataset. They utilize numerous different mathematical methods including the following: inverse distance weighted method, rectangle based blending method, triangle based blending method, finite element based method, Foley's method, global basis function type method, and modified Maude method \cite{franke1982scattered}. All of these techniques focus only on developing a function in order to interpolate scattered data sets.

Later works build off of that approach, by taking classical radial basis functions, such as Duchon's thin plate splines and Hardy's multiquadratics, and compressing them in order to shorten the excessive computation times that result from applying these functions to large data sets, while trying to maintain a smooth data fitting \cite{floater1996multistep}.

There are reapplications of some of these interpolation functions to generate continuous surfaces from irregularly distributed data, in attempts to analyze which function best for spatial analysis. The methods include: inverse square distance method, Kriging method, tension finite difference method, and Hardy's multiquadric method \cite{caruso1998interpolation}. 

Several cases can be found in which these interpolation functions are modified to more accurately apply to specific datasets. One example is the use of a combination of the thin plate smoothing spline and Kringing method in spatial interpolation in order to create a more comprehensive archive of Australian climate data \cite{jeffrey2001using}. Another uses spatial interpolation in improving the MODIS global datasets for terrestrial gross and net primary production \cite{zhao2005improvements}.

Although more applications of function-based interpolation can be found, we could not locate any use of data mining classification models for interpolation purposes.

\section{Methodology}
\subsection{Decision Tree Interpolation}
Decision Tree Interpolation follows this process:
\begin{enumerate}
    \item Build a decision tree from the original data.
    \item For each leaf node in the tree:
    \begin{enumerate}
        \item Obtain all attribute values for associated data instances.
        \item Define ranges for attribute values.
        \begin{itemize}
            \item For numeric attributes, define the range using minimum and maximum values.
            \item For discrete attributes, define the range as all distinct values.
        \end{itemize}
        \item Calculate the distribution of all associated data instances.
        \item Create new data points within the ranges that match the statistical distribution.
    \end{enumerate}
\end{enumerate}
Note that the number of data points created per leaf node is proportional to the number of data points already classified by that leaf node. This ensures that any interpolated data will follow the overall data distribution, at least relative to the data density per leaf node. 

\begin{table*}[!t]
    \caption{Experiment Result Summary}
    \label{table:experiment_result_summary}
    \centering
    \rowcolors{2}{gray!25}{white}
    \begin{tabular}{cccccc}
        \rowcolor{gray!50}
        \textbf{Data Set} & \textbf{Max Tree Depth} & \textbf{OT -> OD} & \textbf{OT -> ND} & \textbf{NT -> OD} & \textbf{NT -> ND} \\
        adult\_sample & 1 & 0.805527123849 & 0.780737704918 & 0.804503582395 & 0.782786885246 \\
        adult\_sample & 2 & 0.808597748209 & 0.809426229508 & 0.249744114637 & 0.813524590164 \\
        adult\_sample & 3 & 0.816786079836 & 0.850102669405 & 0.801432958035 & 0.852156057495 \\
        adult\_sample & 4 & 0.822927328557 & 0.794661190965 & 0.787103377687 & 0.784394250513 \\
        adult\_sample & 5 & 0.822927328557 & 0.84052532833 & 0.792221084954 & 0.84052532833 \\
        car & 1 & 0.700231481481 & 0.710648148148 & 0.700231481481 & 0.710648148148 \\
        car & 2 & 0.777777777778 & 0.783564814815 & 0.774305555556 & 0.789351851852 \\
        car & 3 & 0.824074074074 & 0.809027777778 & 0.824074074074 & 0.815972222222 \\
        car & 4 & 0.894097222222 & 0.903935185185 & 0.889467592593 & 0.915509259259 \\
        car & 5 & 0.96412037037 & 0.966981132075 & 0.938078703704 & 0.982311320755 \\
        iris & 1 & 0.666666666667 & 0.693333333333 & 0.666666666667 & 0.693333333333 \\
        iris & 2 & 0.96 & 0.973333333333 & 0.946666666667 & 0.986666666667 \\
        iris & 3 & 0.973333333333 & 0.959459459459 & 0.953333333333 & 0.972972972973 \\
        iris & 4 & 0.98 & 0.959459459459 & 0.946666666667 & 1.0 \\
        iris & 5 & 1.0 & 1.0 & 0.966666666667 & 1.0 \\
        lung-cancer & 1 & 0.59375 & 0.6 & 0.375 & 0.666666666667 \\
        lung-cancer & 2 & 0.625 & 0.571428571429 & 0.4375 & 0.714285714286 \\
        lung-cancer & 3 & 0.625 & 0.642857142857 & 0.5625 & 1.0 \\
        lung-cancer & 4 & 0.6875 & 0.428571428571 & 0.53125 & 1.0 \\
        lung-cancer & 5 & 0.78125 & 0.538461538462 & 0.53125 & 1.0 \\
        tic\_tac\_toe & 1 & 0.699373695198 & 0.68267223382 & 0.699373695198 & 0.68267223382 \\
        tic\_tac\_toe & 2 & 0.705636743215 & 0.690376569038 & 0.703549060543 & 0.696652719665 \\
        tic\_tac\_toe & 3 & 0.769311064718 & 0.779874213836 & 0.755741127349 & 0.758909853249 \\
        tic\_tac\_toe & 4 & 0.831941544885 & 0.82264957265 & 0.745302713987 & 0.856837606838 \\
        tic\_tac\_toe & 5 & 0.918580375783 & 0.907284768212 & 0.83611691023 & 0.933774834437 \\
        voting & 1 & 0.95632183908 & 0.923766816143 & 0.95632183908 & 0.923766816143 \\
        voting & 2 & 0.95632183908 & 0.956896551724 & 0.95632183908 & 0.956896551724 \\
        voting & 3 & 0.963218390805 & 0.913357400722 & 0.95632183908 & 0.927797833935 \\
        voting & 4 & 0.963218390805 & 0.892156862745 & 0.928735632184 & 0.90522875817 \\
        voting & 5 & 0.972413793103 & 0.866071428571 & 0.937931034483 & 0.895833333333 \\
    \end{tabular}
\end{table*}

\subsection{Interpolated Data Validation}
All interpolated data is validated through this process:
\begin{enumerate}
    \item A new decision tree is built based on the interpolated data. Note that the original data is \textit{not} included here.
    \item Both the new and original decision trees are compared for accuracy against the new and original data sets.
\end{enumerate}
Note that any decision tree with an arbitrarily large maximum depth can classify data with perfect accuracy. Defining a low maximum depth means that classification is imperfect, and it is under these conditions that differences in the quality of different decision trees become apparent.

\subsection{Experiment Parameters}
We completed 30 experiments based on two variables: data source and maximum depth of the decision tree. The maximum depth ranged from one to five, and there were six data sources pulled from Orange's documentation data sets.

\section{Results}
Result data is shown in Table~\ref{table:experiment_result_summary}. The first two columns list the data source and the maximum depth for generated decision trees. Note that two decision trees are generated per row, one for the original data and one for the new (interpolated) data. The last four columns list the accuracy of both the original and new decision trees against the original and new data sets. For example, the column labeled "OT -> ND" shows the accuracy of the original decision tree (OT) when used to classify instances in the new data set (ND). Results and outliers for each of the data sets will be explored further here.
\subsection{adult\_sample}
The only significant outlier in this group was for a max tree depth of two. The "NT -> OD" column shows an accuracy of 0.2497 while the other accuracy values ranged from 0.8085 to 0.8135. Ignoring the outlier, accuracy ranged from 0.7807 to 0.8522 for this data source.
\subsection{car}
Results in this section largely followed expected trends. Accuracy ranged from 0.7002 to 0.9823 for this data source, and accuracy increased in each column as the max tree depth increased.
\subsection{iris}
Results in this section followed expectations, but classification started out relatively inaccurate. At a max tree depth of 1, accuracy ranged from 0.6667 to 0.6933. At a max tree depth of 2, accuracy increased rapidly to range from 0.9467 to 0.9867. Accuracy remained above 0.9467 for all other entries.
\subsection{lung-cancer}
Accuracy had enormous variance here, ranging from 0.375 to 1.0 across all categories. While both decision trees increased in accuracy as max depth increased for their own data sets, they did not consistently increase in accuracy for the other data set.
\subsection{tic\_tac\_toe}
Results in this section were mostly consistent, and accuracy ranged from 0.6827 to 0.9338. Accuracy tended to increase as max tree depth increased, with only slight exceptions.
\subsection{voting}
While there was no clear trend in the results for this data source, accuracy was consistently greater than or equal to 0.8661 for all entries. All but three values were above 0.9, so accuracy was generally high across all instances.

\section{Discussion}
Another section.

\section{Conclusion and Future Work}
Last section.
%\end{document}  % This is where a 'short' article might terminate

%
% The following two commands are all you need in the
% initial runs of your .tex file to
% produce the bibliography for the citations in your paper.
\bibliographystyle{abbrv}
\bibliography{references}  % references.bib is the name of the Bibliography in this case
% You must have a proper ".bib" file
%  and remember to run:
% latex bibtex latex latex
% to resolve all references
%
% ACM needs 'a single self-contained file'!
% That's all folks!
\end{document}
